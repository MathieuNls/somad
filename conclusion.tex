\vspace{0.3cm}
\section{Conclusions and Future Work \label{sec:Conclusions}}
\vspace{0.2cm}
The detection of SOA antipatterns is a crucial activity if we are to ensure
the architectural and overall quality of SBSs. In this paper, we present a new and innovative
approach, SOMAD, for the detection of antipatterns. The approach relies on two complementary
techniques, from two thriving fields in software engineering, mining system traces and software measurement, respectively, both put in an SOA environment.
More precisely, SOMAD detects SOA antipatterns by first discovering strong associations between
services from execution traces and then filtering the resulting knowledge by means of
domain-specific metrics. The usefulness of SOMAD was demonstrated
by applying it to two independently developed SBS. The results of our approach, were compared to
those of its forebear, SODA: The outcome shows that SOMAD is a relevant approach as it is substantially
more precise (by a margin ranging from 2.6\% to 16.67\%) and efficient (2.5+ times faster) while keeping the recall to 100\%. Moreover, SOMAD has a wider coverage than SODA as it can adapt to execution traces from any SOA technology --and is potentially applicable to traces produced by OO systems-- as opposed to a narrow focus on SCA SBSs. 

As a next step, we envision the application of SOMAD in the context of a large data center 
whereby the goal would be to optimize the data center communications. 
In the near future, we shall also investigate alternative mining techniques
to refine our approach with additional information, e.g. directly extracting architectural
overviews with graph pattern mining~\cite{chakrabarti2006graph}
or, detecting recurring patterns of anomalous behavior with rare pattern mining~\cite{fca-rare:szathmary+07}.
Finally, combining explicit semantic representations of SOA antipatterns, e.g. in OWL ontologies, with
powerful mining methods for heterogeneous labeled graphs (see~\cite{onto-pers:adda+10})
seems to be a particularly promising track
for the extraction of complex structural and/or behavioral antipatterns.
