\noindent \emph{Step 1. Metrics Inference:} Metrics to support the interpretation of sequential association rules are inferred from a set of three hypotheses synthesized from the textual description of SOA antipatterns (Table~\ref{tab:List-of-SOA}).
\vspace{0.15cm}
\\
\noindent These hypotheses represent heuristics that enable the identification of \red{architectural} properties relevant to SOA antipatterns. Indeed, after a careful examination of the textual descriptions, we observed that SOA antipatterns can be specified in terms of coupling and cohesion\footnote{Recall coupling basically refers to the degree a services relies on others while cohesion measures the relatedness between its own responsibilities~\cite{stevens1974structured}.}.

%\noindent \textcolor{blue}{In order to infer new metrics able to identify SOA antipatterns properties highlighted from their textual description in ; we need hypotheses helping us to match these properties against the encoded sequential association rules information. These hypothesis work as a kind of heuristic to interpret sequential association rules in order to identify properties relevant to SOA antipatterns. After a careful examination of antipattern textual description, we observe that many of them reflect a high degree of coupling to other services\footnote{The
%coupling basically refers to the degree a services relies on others~\cite{stevens1974structured}.}.
%We therefore conclude that such antipatterns can be defined, at least partially,
%in terms of incoming and outgoing coupling.}



%\noindent To accomplish the detection of antipatterns given in Table~\ref{tab:List-of-SOA} ) we have to decode the sequential association rules information. As described before, a sequential association rule indicates how likely
%is a set of services to be called together in the specific order defined by the rule\footnote{In fact,
%how likely a sequence of services/methods (the rule consequent) is to occur after observing another sequence (the antecedent).}.
%To detect SOA antipatterns within sequential association rules we need to make hypotheses; the key idea behind them is to identify SOA antipatterns properties, such as cohesion or coupling, while analysing sequential association rules rather than service interfaces. 



% in this non-common allegory. Newly created hypothesis are then used to infer metrics able to measure SOA antipatterns properties, for example the cohesion, coupling and number of methods inside the sequential association rules.

\begin{table}
\caption{List of SOA Antipatterns \cite{Moha12-ICSOC-SOASpecificationDetection} \label{tab:List-of-SOA}}
\begin{tabular*}{8.7cm}{@{\extracolsep{\fill}}p{8.7cm}}
\tabularnewline
\textbf{Multi-Service}, \textit{a.k.a} God Object corresponds to a service
that implements a \textbf{multitude of methods} related to different
business and technical abstractions. This aggregate too much into
a single service, such a service is not easily reusable because of
the \textbf{low cohesion} of its methods and is often unavailable
to end-users because of its overload, which may induce a high response
time \cite{Dudney03-J2EEAntipatterns}.\tabularnewline
\noalign{\vskip0.15cm}
\textbf{Tiny Service} is a small service with\textbf{ few methods}, which only
implements part of an abstraction. Such service often requires \textbf{several
coupled services} to be used together, resulting in higher development
complexity and reduced usability. In the extreme case, a Tiny Service
will be limited to \textbf{one method}, resulting in many services
that implement an overall set of requirements \cite{Dudney03-J2EEAntipatterns}.\tabularnewline
\noalign{\vskip0.15cm}
\textbf{Chatty Service} corresponds to a set of services that exchange a \textbf{lot
of small data} of primitive types. The Chatty Service is also characterized
by a\textbf{ high number of method invocations}. Chatty Service chats
a lot with each other \cite{Dudney03-J2EEAntipatterns}.\tabularnewline
\noalign{\vskip0.15cm}
\textbf{The Knot} is a \textbf{set of very} \textbf{low cohesive} services,
which are tightly coupled. These services are thus less reusable.
Due to this complex architecture, the availability of these services
can be low, and their response time high \cite{ArnonSOARotem11-SOAPatterns}.\tabularnewline
\noalign{\vskip0.15cm}
\textbf{Bottleneck Service} is a service that is \textbf{highly used} by other
services or clients. It has a \textbf{high incoming and outgoing coupling}.
Its response time can be higher because it may be used by too many external
clients, for which clients may need to wait to get access to the service.
Moreover, its availability may also be low due to the traffic.\tabularnewline
\noalign{\vskip0.15cm}
\textbf{Service Chain}, a.k.a. Message Chain in OO systems, corresponds
to a \textbf{chain of services}. The Service Chain appears when clients
request \textbf{consecutive service invocations} to fulfill their
goals. This kind of \textbf{dependency chain} reflects the action
of invocation in a transitive manner.
\tabularnewline
\end{tabular*}\end{table}

\begin{conjecture}
If a service A \emph{implies} a service B with a high support and a high confidence, then
A and B are tightly coupled.
\end{conjecture}

\begin{conjecture}
If a service appears in the consequent (antecedent) parts for a high number of associations,
then it has high incoming (outgoing) coupling.
\end{conjecture}

\noindent The above hypotheses qualify the coupling between two specific
services and overall incoming/outgoing coupling. The cohesion is also widely used in
SOA antipattern descriptions. 
%\begin{conjecture}
%\sout{If the number
%of different methods of a given service is
%similar to the number of different partners (Hypothesis 2) it has
%in the service rules,
%then the service is not cohesive.}
%\end{conjecture}
\begin{conjecture}
\rv{1-1} \rv{2-3} \rv{3-1} \red{If the number
of different methods of a service A is
equal or superior to the number of different services invoking A (Hypothesis 2) 
then, the service is not externally cohesive.}
\end{conjecture}

\noindent \red{This definition of cohesion has been introduced by Perepletchikov \textit{et al.}: "\textit{A service is deemed to be Externally cohesive when all of its service operations are invoked by all the clients of this service}" \cite{Perepletchikov2010}}. Based on the above three hypotheses, we have created domain specific metrics
to help us explore the antipattern manifestations that are hidden in the sequential association rules. We use the DSL we defined in~\cite{Moha12-ICSOC-SOASpecificationDetection} to combine them. Metrics are presented in Table~\ref{tab:metrics}. In the
figure, standard mathematical notations are used whenever possible
and extended if necessary. Thus, association rules are visualized
by (X $\rightarrow$ Y) with X and Y represent the antecedent and
the consequent parts, respectively. \emph{K, L} are partner services.
$AR$ stands for the overall set of association rules while $AR_{s}$
and $AR_{m}$ being subsets targeting association rules at service / method level, respectively. \emph{$M_{S}$}
denotes the methods of a given service $S$. Finally, we use non-standard
symbols for sequence operations: $[ ]$ is the sequence constructor,
$\Cup$ stand for append on sequences; $\Subset$ denotes the sub-sequence-of
relationship; and A $\lessdot$ B means the service/method A appears
inside \red{the association rule B}. Metrics can be combined to define other metrics.

\begin{table*}
\caption{Metrics $\Cup$ : append on sequences; $\Subset$ : sub-sequence-of relationship; and A $\lessdot$ B : A appears inside B.\label{tab:metrics}}
{\renewcommand{\arraystretch}{1.5}}
\begin{tabular}{|>{\normalsize}p{2\columnwidth}|}
\hline
\textbf{\textit{Number of Matches (NMA(S))}} : $\#\{X\rightarrow Y\in AR_{s}\mid S\lessdot(X\Cup Y)\}$ \\ 
Follows the number of rules where a service appears, either on the left- or on the right-hand side. \\ \hline \hline
\textbf{\textit{Number of Diff. Partners (NDP(S))}} : $\#\{K\mid X\rightarrow Y\in AR_{s},S\lessdot X,K\lessdot Y\}$
  $+~\#\{K\mid X\rightarrow Y\in AR_{s},S\lessdot Y,K\lessdot X\}$ \\ 
Indicates how many different partners a service
has. Spelled differently, the metric determines whether the
service communicates intensively with surrounding services
or not. \\ \hline \hline
\textbf{\textit{Number of Methods (NM(S))}} : $\#\{K\mid X\rightarrow Y\in AR_{m},K\in M_{s},K\lessdot(X\Cup Y)\}$ \\ 
Counts
the number of occurrences of the methods from a service.
The counting for this metric focuses on method rules.
 \\ \hline \hline 
\textbf{\textit{Cohesion (COH(S))}} : $\frac{NDP(S)}{NM(S)}$ \\ 
Assesses the ratio
between the numbers of partner services and of the available
methods, respectively. \\ \hline \hline
\textbf{\textit{Cross Invocation Dependencies (CID($S_{a},S_{b}$))}} : $\#\{X\rightarrow Y\in AR_{s}\mid S_{a}\lessdot X,S_{b}\lessdot Y\}$ $+~ \#\{X\rightarrow Y\in AR_{s}\mid S_{a}\lessdot Y,S_{b}\lessdot X\}$  \\ 
\red{CID is a keystone of the SOMAD approach}. Indeed, the metric would explore the typical
interactions between services while ignoring less frequent
ones (absent from the mining method output due to the
support threshold). To retrieve this information CID counts
all association rules where a service A (Sa) is present in
the antecedent and a service B (Sb) in the consequent or
\textit{vice versa}. \\ \hline \hline
\textbf{\textit{Incoming Coupling (IC(S))}} : $\sum_{L\in\{K\mid X\rightarrow Y\in AR_{s},K\lessdot X,S\lessdot Y\}}\frac{CID(S,L)}{NDP(S)}$ \\ 
Counts how many times a service is used.
Yet instead of merely counting a unit for each partner service, we
use a contextual value: $\frac{CID(S,X)}{NDP(S)}$ where $X$ is the
partner service. Thus, the larger the portion of the partner service
in the overall number of partners of $S$, the higher the coupling. \\ \hline \hline
\textbf{\textit{Outgoing Coupling (OC(S))}} : $\sum_{L\in\{K\mid X\rightarrow Y\in AR_{s},S\lessdot X,K\lessdot Y\}}\frac{CID(L,S)}{NDP(S)}$ \\ 
The same principle as for \textbf{IC}, yet applied in
a dual manner: counts how many times the argument service
uses other ones. \\ \hline \hline
\textbf{\textit{Transitive Coupling (TC$(S_{a},S_{b})$)}} : $\#\{K\mid X\rightarrow Y\in AR_{s},S_{a}\lessdot X,S_{b}\lessdot Y,({[S_{a},K]\Subset X}\vee[K,S_{b}]\Subset Y)\}$ \\ 
Metric targets the \textit{Service Chain} SOA antipattern 
(see above). First, observe that the founding idea of \textit{Service Chain}
is that absence of direct communication between a pair of services
does not mean zero coupling.
To identify transitive coupling manifestations 
we need to capture the notion of a chain: e.g. a service $S_{a}$ is
in the antecedent of a rule, another one $S_{b}$ is in the consequent of another
rule and both rules are connected by means of a third service $K$
that appears in the consequent of the first rule and in the antecedent
of the second one. Longer chains are possible as well. Thus, in the
basic case, one could have {[}A{]}$\rightarrow${[}B{]} and {[}B{]}$\rightarrow${[}C{]}.
In this configuration, although A and C are not directly coupled, if C
fails, there are good chances that A (and B) would fail too.
\\ \hline 

\end{tabular}
\end{table*}
