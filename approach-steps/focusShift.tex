\noindent \emph{Focus shift.} This feature is the main reason for SOMAD performing better than SODA in the identification of truly harmful SOA antipatterns. \rv{2-6}
\vspace{.15cm}
\\
\noindent Observe that SOMAD hypotheses shift the focus of the antipattern search from pure architectural considerations to usage, thus neglecting the exact values of some basic metrics. It is a natural choice since SOMAD does not access exact values through service interfaces or implementation. Moreover, analyzing a system from the usage view angle should --and this was proven by our experimental study (see below)-- result in a better precision. Consider a service named \emph{Half-Deprecated Service} composed of four methods: A, B, C and D. Assume the methods C and D are outdated yet the service still exposes them to ensure retro-compatibility. One way to compute the cohesion of our service is to count how many of its methods are used during a session by a unique user. Since half of the methods are outdated it is highly probable that any user will consume at most the other half. Therefore, if cohesion is computed from the service interface, it would amount to 0.5 (2/4) which should raise the suspicions of low-cohesion SOA antipatterns. In contrast, if the cohesion is computed from execution traces the result will tend to be 1.0. Indeed, the unforeseen calls to the deprecated methods will most probably be discarded due to their their low support in the execution traces. In summary,  because of its focus on usage, SOMAD should perform better than SODA in detecting harmful SOA antipatterns.