\subsection{Introduction to Sequential Association Rule Mining}

\noindent In the data mining field, ARM is a
well-established method for discovering co-occurrences between attributes
in the objects of a large data set~\cite{piatetsky1991discovery}. Plain
associations have the form $X \rightarrow Y$, where $X$ and $Y$,
called the \textit{antecedent} and the \textit{consequent}, respectively, are sets of descriptors
(purchases by a customer, network alarms, or any other general kind of events).
Even though plain association rules could serve some relevant information, we are interested here in
the sequences of service invocations.
We therefore adopt a variant called sequential association rules in which
both $X$ and $Y$ become sequences of descriptors.
Moreover, our sequences follow a temporal order with the antecedent preceding the consequent. 
Rules of this type mined from traces reveal crucial
information about the likelihood that services appear together in an execution
trace and, more importantly, in a specific order. 
For instance, a strong rule \emph{ServiceA $,$ ServiceB } implies \emph{ServiceC} would mean that after executing A and then B, there are good chances to see C in the trace.
The conciseness of this example should not confuse the reader as in practical cases
the sequences appearing in a rule can be of an arbitrary length.
Furthermore, the strength of the rule is measured by the \textit{confidence} metric: In probabilistic terms,
it measures the conditional probability of C appearing down the line.
Beside that, the significance of a rule, i.e. how many times it appears in the data,
is provided by its \textit{support} measure.
To ensure only rules of potentially high interestingness are mined,
the mining task is tuned by minimal thresholds to
output only the sufficiently high scores for both metrics.
