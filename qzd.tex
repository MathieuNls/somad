\begin{figure*}
{\renewcommand{\arraystretch}{1.5}
\begin{tabular}{>{\small}p{2\columnwidth}
\hline
\textbf{\textit{Number of Matches (NMA(S))}} : $\#\{X\rightarrow Y\in AR_{s}\mid S\lessdot(X\Cup Y)\}$ \\
Follows the number of rules where a service appears, either on the left- or on the right-hand side. \\ \hline \hline
\textbf{\textit{Number of Diff. Partners (NDP(S))}} : $\#\{K\mid X\rightarrow Y\in AR_{s},S\lessdot X,K\lessdot Y\}$
  $+~\#\{K\mid X\rightarrow Y\in AR_{s},S\lessdot Y,K\lessdot X\}$ \\ 
Indicates how many different partners a service
has. Spelled differently, the metric determines whether the
service communicates intensively with surrounding services
or not. \\ \hline \hline
\textbf{\textit{Number of Methods (NM(S))}} : $\#\{K\mid X\rightarrow Y\in AR_{m},K\in M_{s},K\lessdot(X\Cup Y)\}$ \\ 
Counts
the number of occurrences of the methods from a service.
The counting for this metric focuses on method rules.
 \\ \hline \hline 
\textbf{\textit{Cohesion (COH(S))}} : $\frac{NDP(S)}{NM(S)}$ \\
Assesses the ration
between the numbers of partner services and of the available
methods, respectively. \\ \hline \hline
\textbf{\textit{Cross Invocation Dependencies (CID($S_{a},S_{b}$))}} : $\#\{X\rightarrow Y\in AR_{s}\mid S_{a}\lessdot X,S_{b}\lessdot Y\}$ $+~ \#\{X\rightarrow Y\in AR_{s}\mid S_{a}\lessdot Y,S_{b}\lessdot X\}$  \\
\red{CID is a keystone of the SOMAD approach}. Indeed, the metric would explore the typical
interactions between services while ignoring less frequent
ones (absent from the mining method output due to the
support threshold). To retrieve this information CID counts
all association rules where a service A (Sa) is present in
the antecedent and a service B (Sb) in the consequent or
vice versa. \\ \hline \hline
\textbf{\textit{Incoming Coupling (IC(S))}} : $\sum_{L\in\{K\mid X\rightarrow Y\in AR_{s},K\lessdot X,S\lessdot Y\}}\frac{CID(S,L)}{NDP(S)}$ \\
Count how many times a service is used.
Yet instead of merely counting a unit for each partner service, we
use a contextual value: $\frac{CID(S,X)}{NDP(S)}$ where $X$ is the
partner service. Thus, the larger the portion of the partner service
in the overall number of partners of $S$, the higher the coupling. \\ \hline \hline
\textbf{\textit{Outgoing Coupling (OC(S))}} : $\sum_{L\in\{K\mid X\rightarrow Y\in AR_{s},S\lessdot X,K\lessdot Y\}}\frac{CID(L,S)}{NDP(S)}$ \\
The same principle as for \textbf{IC} applies, yet in
a dual manner. count how many times the argument service
uses other services. \\ \hline \hline
\textbf{\textit{Transitive Coupling (TC$(S_{a},S_{b})$)}} : $\#\{K\mid X\rightarrow Y\in AR_{s},S_{a}\lessdot X,S_{b}\lessdot Y,({[S_{a},K]\Subset X}\vee[K,S_{b}]\Subset Y)\}$ \\
Is
aimed at spotting a common antipattern in SOA, the \textit{Service Chain}
(see above). First, observe that the fact that a pair of services
does not communicate directly does not mean that they are not coupled:
that is the founding idea of \textit{Service Chain} and of our metric too.
To identify transitive coupling manifestations within the extracted
association rules, which is more challenging than the direct coupling,
we need to capture describing a chain: e.g. a service $S_{a}$ is
in the antecedent of a rule, $S_{b}$ is in the consequent of another
rule and both rules are connected by means of a third service $K$
that appears in the consequent of the first rule and in the antecedent
of the second one. Longer chains are possible as well. Thus, in the
basic case, we could have {[}A{]}$\rightarrow${[}B{]} and {[}B{]}$\rightarrow${[}C{]},
with this configuration, A and C are not directly coupled but if C
fails, there are good chances that A and B would fail too. 
\end{tabular}}
\caption{Metrics $\Cup$ : append on sequences; $\Subset$ : sub-sequence-of relationship; and A $\lessdot$ B : A appears inside B.\label{tab:metrics}}
\end{figure*}
