
\section{The SOMAD Approach \label{sec:Methodology}}



We propose a five step approach, named SOMAD (Service Oriented Mining for Antipatterns Detection), for the detection of SOA antipatterns within execution traces of SBSs. This new approach is a variant of SODA based on execution traces, which may come from any kind of SBSs. 
In contrast, SODA applies specifically on SCA SBSs using a set of scenarios and SCA-based techniques. In particular, in SOMAD, we specify a new set of metrics that apply to sequential association rules mined on execution traces whereas, in SODA, metrics apply to the concrete invocation of SBSs' interfaces using a set of scenarios. 
%As a result, the detection of SOA antipatterns with SOMAD is non-intrusive and technology agnostic because it only requires execution traces.
Figure \ref{fig:The-SODA-approach} shows an overview of SOMAD. We emphasized in grey the two new steps specific to SOMAD and added to the SODA approach. \textit{Step~1. Metric Inference} is supported by the creation of a set of hypotheses made from the textual description of SOA antipatterns. The hypotheses underlie the definition of new  metrics to support the interpretation of association rules. \textit{Step~4. Association Rule Mining} (ARM) discovers interesting sequential associations in execution traces of the targeted SBS. \rv{2-2} Output sequential association rules represent \red{statistically} interesting relations between services inside traces. In what follows, we first introduce key concepts of sequential ARM and then, present the overall process of SOMAD. Finally, we provide some implementation details.

\subsection{Introduction to Sequential Association Rule Mining}

\noindent In the data mining field, ARM is a
well-established method for discovering co-occurrences between attributes
in the objects of a large data set~\cite{piatetsky1991discovery}. Plain
associations have the form $X \rightarrow Y$, where $X$ and $Y$,
called the \textit{antecedent} and the \textit{consequent}, respectively, are sets of descriptors
(purchases by a customer, network alarms, or any other general kind of events).
Even though plain association rules could serve some relevant information, we are interested here in
the sequences of service invocations.
We therefore adopt a variant called sequential association rules in which
both $X$ and $Y$ become sequences of descriptors.
Moreover, our sequences follow a temporal order with the antecedent preceding the consequent. 
Rules of this type mined from traces reveal crucial
information about the likelihood that services appear together in an execution
trace and, more importantly, in a specific order. 
For instance, a strong rule \emph{ServiceA $,$ ServiceB } implies \emph{ServiceC} would mean that after executing A and then B, there are good chances to see C in the trace.
The conciseness of this example should not confuse the reader as in practical cases
the sequences appearing in a rule can be of an arbitrary length.
Furthermore, the strength of the rule is measured by the \textit{confidence} metric: In probabilistic terms,
it measures the conditional probability of C appearing down the line.
Beside that, the significance of a rule, i.e. how many times it appears in the data,
is provided by its \textit{support} measure.
To ensure only rules of potentially high interestingness are mined,
the mining task is tuned by minimal thresholds to
output only the sufficiently high scores for both metrics.



\subsection{SOMAD Process}


\noindent \emph{Step 1. Metrics Inference:} Metrics to support the interpretation of sequential association rules are inferred from a set of three hypotheses synthesized from the textual description of SOA antipatterns (Table~\ref{tab:List-of-SOA}).
\vspace{0.15cm}
\\
\noindent These hypotheses represent heuristics that enable the identification of \red{architectural} properties relevant to SOA antipatterns. Indeed, after a careful examination of the textual descriptions, we observed that SOA antipatterns can be specified in terms of coupling and cohesion\footnote{Recall coupling basically refers to the degree a services relies on others while cohesion measures the relatedness between its own responsibilities~\cite{stevens1974structured}.}.

%\noindent \textcolor{blue}{In order to infer new metrics able to identify SOA antipatterns properties highlighted from their textual description in ; we need hypotheses helping us to match these properties against the encoded sequential association rules information. These hypothesis work as a kind of heuristic to interpret sequential association rules in order to identify properties relevant to SOA antipatterns. After a careful examination of antipattern textual description, we observe that many of them reflect a high degree of coupling to other services\footnote{The
%coupling basically refers to the degree a services relies on others~\cite{stevens1974structured}.}.
%We therefore conclude that such antipatterns can be defined, at least partially,
%in terms of incoming and outgoing coupling.}



%\noindent To accomplish the detection of antipatterns given in Table~\ref{tab:List-of-SOA} ) we have to decode the sequential association rules information. As described before, a sequential association rule indicates how likely
%is a set of services to be called together in the specific order defined by the rule\footnote{In fact,
%how likely a sequence of services/methods (the rule consequent) is to occur after observing another sequence (the antecedent).}.
%To detect SOA antipatterns within sequential association rules we need to make hypotheses; the key idea behind them is to identify SOA antipatterns properties, such as cohesion or coupling, while analysing sequential association rules rather than service interfaces. 



% in this non-common allegory. Newly created hypothesis are then used to infer metrics able to measure SOA antipatterns properties, for example the cohesion, coupling and number of methods inside the sequential association rules.

\begin{table}
\caption{List of SOA Antipatterns \cite{Moha12-ICSOC-SOASpecificationDetection} \label{tab:List-of-SOA}}
\begin{tabular*}{8.7cm}{@{\extracolsep{\fill}}p{8.7cm}}
\tabularnewline
\textbf{Multi-Service}, \textit{a.k.a} God Object corresponds to a service
that implements a \textbf{multitude of methods} related to different
business and technical abstractions. This aggregate too much into
a single service, such a service is not easily reusable because of
the \textbf{low cohesion} of its methods and is often unavailable
to end-users because of its overload, which may induce a high response
time \cite{Dudney03-J2EEAntipatterns}.\tabularnewline
\noalign{\vskip0.15cm}
\textbf{Tiny Service} is a small service with\textbf{ few methods}, which only
implements part of an abstraction. Such service often requires \textbf{several
coupled services} to be used together, resulting in higher development
complexity and reduced usability. In the extreme case, a Tiny Service
will be limited to \textbf{one method}, resulting in many services
that implement an overall set of requirements \cite{Dudney03-J2EEAntipatterns}.\tabularnewline
\noalign{\vskip0.15cm}
\textbf{Chatty Service} corresponds to a set of services that exchange a \textbf{lot
of small data} of primitive types. The Chatty Service is also characterized
by a\textbf{ high number of method invocations}. Chatty Service chats
a lot with each other \cite{Dudney03-J2EEAntipatterns}.\tabularnewline
\noalign{\vskip0.15cm}
\textbf{The Knot} is a \textbf{set of very} \textbf{low cohesive} services,
which are tightly coupled. These services are thus less reusable.
Due to this complex architecture, the availability of these services
can be low, and their response time high \cite{ArnonSOARotem11-SOAPatterns}.\tabularnewline
\noalign{\vskip0.15cm}
\textbf{Bottleneck Service} is a service that is \textbf{highly used} by other
services or clients. It has a \textbf{high incoming and outgoing coupling}.
Its response time can be higher because it may be used by too many external
clients, for which clients may need to wait to get access to the service.
Moreover, its availability may also be low due to the traffic.\tabularnewline
\noalign{\vskip0.15cm}
\textbf{Service Chain}, a.k.a. Message Chain in OO systems, corresponds
to a \textbf{chain of services}. The Service Chain appears when clients
request \textbf{consecutive service invocations} to fulfill their
goals. This kind of \textbf{dependency chain} reflects the action
of invocation in a transitive manner.
\tabularnewline
\end{tabular*}\end{table}

\begin{conjecture}
If a service A \emph{implies} a service B with a high support and a high confidence, then
A and B are tightly coupled.
\end{conjecture}

\begin{conjecture}
If a service appears in the consequent (antecedent) parts for a high number of associations,
then it has high incoming (outgoing) coupling.
\end{conjecture}

\noindent The above hypotheses qualify the coupling between two specific
services and overall incoming/outgoing coupling. The cohesion is also widely used in
SOA antipattern descriptions. 
%\begin{conjecture}
%\sout{If the number
%of different methods of a given service is
%similar to the number of different partners (Hypothesis 2) it has
%in the service rules,
%then the service is not cohesive.}
%\end{conjecture}
\begin{conjecture}
\rv{1-1} \rv{2-3} \rv{3-1} \red{If the number
of different methods of a service A is
equal or superior to the number of different services invoking A (Hypothesis 2) 
then, the service is not externally cohesive.}
\end{conjecture}

\noindent \red{This definition of cohesion has been introduced by Perepletchikov \textit{et al.}: "\textit{A service is deemed to be Externally cohesive when all of its service operations are invoked by all the clients of this service}" \cite{Perepletchikov2010}}. Based on the above three hypotheses, we have created domain specific metrics
to help us explore the antipattern manifestations that are hidden in the sequential association rules. We use the DSL we defined in~\cite{Moha12-ICSOC-SOASpecificationDetection} to combine them. Metrics are presented in Table~\ref{tab:metrics}. In the
figure, standard mathematical notations are used whenever possible
and extended if necessary. Thus, association rules are visualized
by (X $\rightarrow$ Y) with X and Y represent the antecedent and
the consequent parts, respectively. \emph{K, L} are partner services.
$AR$ stands for the overall set of association rules while $AR_{s}$
and $AR_{m}$ being subsets targeting association rules at service / method level, respectively. \emph{$M_{S}$}
denotes the methods of a given service $S$. Finally, we use non-standard
symbols for sequence operations: $[ ]$ is the sequence constructor,
$\Cup$ stand for append on sequences; $\Subset$ denotes the sub-sequence-of
relationship; and A $\lessdot$ B means the service/method A appears
inside \red{the association rule B}. Metrics can be combined to define other metrics.

\begin{table*}
\caption{Metrics $\Cup$ : append on sequences; $\Subset$ : sub-sequence-of relationship; and A $\lessdot$ B : A appears inside B.\label{tab:metrics}}
{\renewcommand{\arraystretch}{1.5}}
\begin{tabular}{|>{\normalsize}p{2\columnwidth}|}
\hline
\textbf{\textit{Number of Matches (NMA(S))}} : $\#\{X\rightarrow Y\in AR_{s}\mid S\lessdot(X\Cup Y)\}$ \\ 
Follows the number of rules where a service appears, either on the left- or on the right-hand side. \\ \hline \hline
\textbf{\textit{Number of Diff. Partners (NDP(S))}} : $\#\{K\mid X\rightarrow Y\in AR_{s},S\lessdot X,K\lessdot Y\}$
  $+~\#\{K\mid X\rightarrow Y\in AR_{s},S\lessdot Y,K\lessdot X\}$ \\ 
Indicates how many different partners a service
has. Spelled differently, the metric determines whether the
service communicates intensively with surrounding services
or not. \\ \hline \hline
\textbf{\textit{Number of Methods (NM(S))}} : $\#\{K\mid X\rightarrow Y\in AR_{m},K\in M_{s},K\lessdot(X\Cup Y)\}$ \\ 
Counts
the number of occurrences of the methods from a service.
The counting for this metric focuses on method rules.
 \\ \hline \hline 
\textbf{\textit{Cohesion (COH(S))}} : $\frac{NDP(S)}{NM(S)}$ \\ 
Assesses the ratio
between the numbers of partner services and of the available
methods, respectively. \\ \hline \hline
\textbf{\textit{Cross Invocation Dependencies (CID($S_{a},S_{b}$))}} : $\#\{X\rightarrow Y\in AR_{s}\mid S_{a}\lessdot X,S_{b}\lessdot Y\}$ $+~ \#\{X\rightarrow Y\in AR_{s}\mid S_{a}\lessdot Y,S_{b}\lessdot X\}$  \\ 
\red{CID is a keystone of the SOMAD approach}. Indeed, the metric would explore the typical
interactions between services while ignoring less frequent
ones (absent from the mining method output due to the
support threshold). To retrieve this information CID counts
all association rules where a service A (Sa) is present in
the antecedent and a service B (Sb) in the consequent or
\textit{vice versa}. \\ \hline \hline
\textbf{\textit{Incoming Coupling (IC(S))}} : $\sum_{L\in\{K\mid X\rightarrow Y\in AR_{s},K\lessdot X,S\lessdot Y\}}\frac{CID(S,L)}{NDP(S)}$ \\ 
Counts how many times a service is used.
Yet instead of merely counting a unit for each partner service, we
use a contextual value: $\frac{CID(S,X)}{NDP(S)}$ where $X$ is the
partner service. Thus, the larger the portion of the partner service
in the overall number of partners of $S$, the higher the coupling. \\ \hline \hline
\textbf{\textit{Outgoing Coupling (OC(S))}} : $\sum_{L\in\{K\mid X\rightarrow Y\in AR_{s},S\lessdot X,K\lessdot Y\}}\frac{CID(L,S)}{NDP(S)}$ \\ 
The same principle as for \textbf{IC}, yet applied in
a dual manner: counts how many times the argument service
uses other ones. \\ \hline \hline
\textbf{\textit{Transitive Coupling (TC$(S_{a},S_{b})$)}} : $\#\{K\mid X\rightarrow Y\in AR_{s},S_{a}\lessdot X,S_{b}\lessdot Y,({[S_{a},K]\Subset X}\vee[K,S_{b}]\Subset Y)\}$ \\ 
Metric targets the \textit{Service Chain} SOA antipattern 
(see above). First, observe that the founding idea of \textit{Service Chain}
is that absence of direct communication between a pair of services
does not mean zero coupling.
To identify transitive coupling manifestations 
we need to capture the notion of a chain: e.g. a service $S_{a}$ is
in the antecedent of a rule, another one $S_{b}$ is in the consequent of another
rule and both rules are connected by means of a third service $K$
that appears in the consequent of the first rule and in the antecedent
of the second one. Longer chains are possible as well. Thus, in the
basic case, one could have {[}A{]}$\rightarrow${[}B{]} and {[}B{]}$\rightarrow${[}C{]}.
In this configuration, although A and C are not directly coupled, if C
fails, there are good chances that A (and B) would fail too.
\\ \hline 

\end{tabular}
\end{table*}


\vspace{0.10cm}

\noindent \emph{Step 2. Specification of SOA Antipatterns:} The combination of metrics defined in the previous step allows the specification of SOA antipatterns in the form of sets of rules, called \textit{rule cards}.
\vspace{0.15cm}
\\
\noindent For the individual
metrics and combinations thereof, the values that trigger the detailed
examination of a case are not fixed beforehand. Instead, we use a
boxplot-based statistical technique that exploits the distribution
of all values across the sets of services, methods, and rules. Moreover,
the computed values are further weighted using the quality metrics
for associations, i.e. support and confidence, so that the strongest
rules could be favored. The \textit{rule cards} used to specify SOA  antipatterns are presented in Figure~\ref{fig:rules}. As an example, the rule card corresponding to the Tiny Service specification (Figure 3(b)) is composed of three rules. The first one (line 2) is the intersection of two rules (lines 3, 4), which define two metrics: a high Outgoing Coupling (OC) and a low Number of Method (NM).
%We determine if a value it's high or not with the boxplot statistical
%method. A boxplot distinguish differences between population of values
%using their statistical distribution. Thus, using this technique,
%we don't have to set subjective thresholds to detect antipatterns.
%Moreover, the values are weighted by $\frac{support}{confidence}$
%for highlighting most confident and supported association rules.
%The rule cards used to detect the individual antipatterns are as
%follows. %We now present how these metrics based on our conjectures can match
%%the textual description of SOA antipatterns.
%\textit{Multi-service} is signaled whenever high values of \textbf{NMA} and
%\textbf{NM} co-occur with low values of \textbf{COH}. For \textit{Tiny service}, high
%values of \textbf{OC} with low values of \textbf{NM} are indicative. Very high values
%of \textbf{NMA} and \textbf{NPD} (i.e. outliers in the boxplot), correspond to the \textit{Chatty service} while \textit{Bottleneck} is characterized
%by high values of both \textbf{IC} and \textbf{OC}.  In addition of antipatterns detection describe earlier, we can use our two most complex metrics--\textbf{CID} and \textbf{TC}--to detect other SOA antipatterns. The \textit{Knot} typically involves low values of \textbf{COH}
%combined with high values of \textbf{CID}. Finally,  \textit{Service chain} corresponds
%to the maximal distance between services. Here the distance reflects the (maximal) number of calls necessary to reach one of the services from the other one and it is measured by \textbf{TC}.

\begin{figure}
\begin{small}
\scriptsize
1~RULE\_CARD:~\emph{\textbf{MultiService}}~\{\\
2~~RULE:~\emph{\textbf{MultiService}}\{INTER~\emph{\textbf{LowCohesion}}~\emph{\textbf{ManyMethods}}~\emph{\textbf{ManyMatches}}\};\\
3~~RULE:~\emph{\textbf{LowCohesion}}\{COH~LOW\};\\
4~~RULE:~\emph{\textbf{ManyMethods}}\{NM~HIGH\};\\
5~~RULE:~\emph{\textbf{ManyMatches}}\{NMA~HIGH\};\\
6~\};
\vspace{-0.2cm}
\begin{center}
(a) Multi Service
\end{center}
\vspace{-0.2cm}
1~RULE\_CARD:~\emph{\textbf{TinyService}}~\{\\
2~~RULE:~\emph{\textbf{TinyService}}\{INTER~\emph{\textbf{HighOutgoingCoupling}}~\emph{\textbf{FewMethods}}\};\\
3~~RULE:~\emph{\textbf{HighOutgoingCoupling}}\{OC~HIGH\};\\
4~~RULE:~\emph{\textbf{FewMethods}}\{NM~LOW\};\\
5~\};
\vspace{-0.2cm}
\begin{center}
(b) Tiny Service
\end{center}
\vspace{-0.2cm}
1~RULE\_CARD:~\emph{\textbf{ChattyService}}~\{\\
2~~RULE:~\emph{\textbf{ChattyService}}\{INTER~\emph{\textbf{ManyPartners}}~\emph{\textbf{ManyMatches}}\};\\
3~~RULE:~\emph{\textbf{ManyPartners}}\{NDP~VERY HIGH\};\\
4~~RULE:~\emph{\textbf{ManyMatches}}\{NMA~VERY HIGH\};\\
5~\};
\vspace{-0.2cm}
\begin{center}
(c) Chatty Service
\end{center}
\vspace{-0.2cm}
1~RULE\_CARD:~\emph{\textbf{BottleNeck}}~\{\\
2~~RULE:~\emph{\textbf{BottleNeck}}\{INTER~\emph{\textbf{HighOutgoingCoupling}}~\emph{\textbf{HighIncomingCoupling}}\};\\
3~~RULE:~\emph{\textbf{HighOutgoingCoupling}}\{OC~HIGH\};\\
4~~RULE:~\emph{\textbf{HighIncomingCoupling}}\{IC~HIGH\};\\
5~\};
\vspace{-0.2cm}
\begin{center}
(d) BottleNeck Service
\end{center}
\vspace{-0.2cm}
1~RULE\_CARD:~\emph{\textbf{KnotService}}~\{\\
2~~RULE:~\emph{\textbf{KnotService}}\{INTER~\emph{\textbf{LowCohesion}}~\emph{\textbf{HighCrossInvocation}}\};\\
3~~RULE:~\emph{\textbf{LowCohesion}}\{COH~LOW\};\\
4~~RULE:~\emph{\textbf{HighCrossInvocation}}\{CID~HIGH\};\\
5~\};
\vspace{-0.2cm}
\begin{center}
(e) Knot Service
\end{center}
\vspace{-0.2cm}
1~RULE\_CARD:~\emph{\textbf{ServiceChain}}~\{\\
2~~RULE:~\emph{\textbf{ServiceChain}}\{\emph{\textbf{HighTransitiveCoupling}}\};\\
3~~RULE:~\emph{\textbf{HighTransitiveCoupling}}\{TC~HIGH\};\\
4~\};
\vspace{-0.2cm}
\begin{center}
(f) Service Chain
\end{center}
\end{small}
\vspace{-0.2cm}
\caption{Rule Cards\label{fig:rules}}
\end{figure}

%The first metric, named \textbf{\textit{NMA}}
%(Number of MAtch), follows the number of rules where a service appears,
%either on the left- or on the right-hand side. \textbf{\textit{NDP}}
%(Number of different Partners) indicates how many different partners
%a service has. Spelled differently, the metric determines whether
%the service communicates intensively with surrounding services or
%not. The next pair of metrics is in charge of assessing the coupling,
%or, more precisely, what we call the \textit{incoming coupling} and
%the \textit{outgoing one}. The basic intuition behind \textbf{\textit{IC}}\textit{(S)}
%(Incoming coupling) is to count how many times a service is used.
%Yet instead of merely counting a unit for each partner service, we
%use a contextual value: $\frac{CID(S,X)}{NDP(S)}$ where $X$ is the
%partner service. Thus, the larger the portion of the partner service
%in the overall number of partners of $S$, the higher the coupling.
%For the outgoing variant of the metric, \textbf{\textit{OC}} (Outgoing
%coupling), the same principle applies, yet in a dual manner. count
%how many times the argument service uses other services. Next, \textbf{\textit{NM}} (Number of Methods) counts the number
%of occurrences of the methods from a service. The counting for this
%metric focuses on method rules. Finally, our last metric, called \textbf{\textit{COH}} for Cohesion, assesses the ration between the numbers of partner services and of the available methods, respectively. 

\vspace{0.10cm}
\noindent \emph{Step 3. Generation of Detection Algorithms:}  This step stays unchanged from SODA, as described in Section~\ref{SODA}.

\noindent \emph{\\Step 4. Association Rule Mining:} Execution traces are analyzed to extract the sequential association rules.
\vspace{0.15cm}
\\
\noindent Association rules are extracted from
a collection of sequence-shaped transactions 
with respect to a minimal support and a minimal confidence threshold.  A transaction is a time-ordered set of different services and method calls. 
Recall that the support of a pattern, i.e. sequence of items (services or service methods),
reflects the overall percentage of transactions that contain the pattern,
whereas the confidence measures the likelihood of the consequent following
the occurrence of the antecedent in a transaction.
For our experiments (see next section) we set the values of
the thresholds to  40\% and 60\%, respectively.
The choice of these values does not follow any
specific indication, general law from ARM or deeper insight
into the SBS architecture.
As our approach is at its exploratory stage,
we were only guided by the need to filter out all spurious
associations while still keeping enough rules to represent the
most significant calls (regulated via the support threshold).
Moreover, we needed enough confidence in the threshold to make appear
the most significant alternatives (rule consequent) for the termination of
a specific sequence of calls (rule antecedent) while suppressing the
less significant ones.  \rv{2-4}
\red{Thus, we have made several incremental attempts, starting from 10\% and 40\% respectively for the support and the confidence. For each attempt, we modified one of the two values by 5\% and observed the number of generated rules.} The current values seem to offer the best trade-off between size and completeness of scenarios. Now we faced a two-fold possibility for the effective ARM method
to use on our traces. In fact, most sequential pattern mining and ARM algorithms
have been designed for structures that are slightly more general than ours, i.e.
involving sequences of \textit{sets} (instead of single items).
Efficient sequential pattern/rule miners have been published, e.g. the PrefixSpan method~\cite{pei2004mining}.
In contrast, execution traces do not compile to fully-blown sequential transactions as the underlying structures are mere sequences of singletons, a data format known for at least 15 years yet rarely exploited by the data mining community, arguably because it is less challenging to mine. However, many practical applications have been reported where such data arise, inclusive software log mining (see Section~\ref{sec:related-work}). In the general data mining literature, mining from pure sequences, as opposed to sequences made of sets, has been addressed under the name of episode mining~\cite{DMKD}. Episodes are made of \textit{events} and in a sense, service calls are events. Arguably the largest body of knowledge on the subject belongs to the web usage mining field: The input data is again a system trace, yet this time the trace of requests sent to a web server~\cite{pei2000mining}.
Since sequential patterns are more general than the pure sequence ones, mining algorithms designed for the former might prove to be less efficient when applied to the latter (as additional steps might be required for listing all significant sets).
\rv{1-2} \red{Nevertheless, to jump-start our experimental study and given the specificity of our datasets, we choose the RuleGrowth algorithm \cite{fournier2011rulegrowth} that seemed to fit at best}. Although it has not been optimized for pure sequences its performances are more than satisfactory. In summary, at the end of this, we have extracted the \red{statistically} relevant relationship between services in the form of sequential association rules.

%\vspace{-0.1cm}
\noindent \emph{Step 5. Detection of SOA Antipatterns}
\vspace{0.1cm}
\\
\noindent The last step of SOMAD applies the detection algorithms generated in Step 3 to the sequential association rules mined in Step 4. At the end of this step, services in the SBS suspected of being involved in an antipattern are identified and stored for further examination.

\subsection{Implementation Details\label{details}}

\noindent In this subsection, we present implementation details for other steps that may support the SOMAD approach. 

\vspace{.15cm}
\noindent \emph{Generation of Execution Traces.} In case execution traces are not available, this step allows their generation.
\vspace{.15cm}
\\
\noindent If the target SBS does not produce qualitative execution traces that contain all the required information, we have to instrument it. \rv{3-2} \red{Thus, SOMAD requires either the its source code or the execution environment.}
%Log files are composed of line that refer to a unique point in the target SBS. For example, a log file containing every method call requires to log statement--two lines--one when entering and one when exiting the methods. 
In fact, such traces enable low-tech application debugging support whenever
debuggers are unavailable or inapplicable (frequently the case with
SOA environments). Therefore, even if 
trace producing can introduce source code obfuscation, it may nevertheless have some secondary benefits
e.g. in terms of design quality as the code must be well mastered in order to correctly instrument.
%logs
%can reduce the understandability of source code while they add many lines
%but in the other hand, these extra statements constrain developer
%to really understand processes they are building, and doubtlessly avoid bad design
%during the development stage.
This technique of tracing is the most common. If the source code is unavailable an alternative consists in instrumenting the running environment of the SBS, i.e.  the virtual machine, the web server, or the operating system. For example, LTTng \cite{Fournier2009} instruments Linux to produce traces with a very low overhead.

%Performance issues can also enter
%into consideration. Indeed printing these thousands of statement
%into log files is resources consuming. Therefore, we made our logging
%system capable of deactivating in a way inspired by Apache
%configuration files.

To ease automated processing of traces, we provide a template (see Figure \ref{fig:Logs-shape}) that is a good trade-off between simplicity and information content. In this template, a method invocation generates two lines, an opening and a closing one with belonging customer identification (IP address) and a timestamp.

%SBSs often contain built-in tracing systems, however, these built-in mechanisms can be very distant from our template. Thus, we made SOMAD adaptable by a simple--regular expression based--domain specific language (Figure \ref{fig:Logs-shape}). Consequently, traces can be organized in various orders.

\begin{figure}[h]
\framebox{\begin{minipage}[t]{1\columnwidth -0.4cm}%
IP timestamp void methodA.ServiceA();\\
	~~~~	IP~~~~timestamp void methodB.ServiceB();\\
	~~~~IP~~~~timestamp end void methodB.ServiceB();\\
IP timestamp end void methodA.ServiceA();
%\\
%time\{\textasciicircum\textbackslash w+\textbackslash s\textbackslash d\textbackslash d\textbackslash s\textbackslash :\textbackslash d\textbackslash d:\textbackslash d\textbackslash d.\textbackslash d+\} || end\{end\}\\
%method\{\textasciicircum[\textasciicircum.]*.(.*)\$\} ||
%service\{\$[\textasciicircum.]*.(*.)\}\\
%customer\{\textbackslash b\textbackslash d\{1,3\}\textbackslash .\textbackslash d\{1,3\}\textbackslash .\textbackslash d\{1,3\}\textbackslash .\textbackslash d\{1, 3\}\textbackslash b\}\\
%line\{ *customer *time *(end)? *method.service *\}
\end{minipage}}
\caption{Trace template\label{fig:Logs-shape}}
\end{figure}
\vspace{.15cm}
\vspace{.20cm}
\noindent \emph{Collecting and Aggregating Traces.} The goal here is to download all distributed trace files and merge them into a single one.
\vspace{.15cm}
\\
\noindent Traces are typically generated by a set of services within the SBS. Their collection and aggregation is a key yet non-trivial task~\cite{Wilde2008}. Indeed, the dynamic and distributed nature of SBSs is the origin of some serious challenges. One of them is related to the distribution of SBSs and, hence, of execution traces. In fact, each  service will generate its execution traces in its own running environment. Therefore, we need to know the name and running place for each service and to have a mechanism for download / retrieval of execution traces on each running environment. Moreover, services can be consumed by several customers simultaneously, hence execution traces can be interleaved. To solve these problem we applied an approach inspired by A. Yousefi and K. Startipi~\cite{Yousefi2011}: We first gather all executions log files in one file. Then, we sort execution traces using their \rv{2-5} timestamps and exploit the caller-callee relationships determined by service and method names to identify blocks of concurrent traces.
%\begin{itemize}
%\item Download distributed execution logs from each different location using our prior knowledge (usually a service-distant repository relationship).
%\item Gather all downloaded execution logs in a single file.
%\item Sort execution traces according to their time stamp; this one requires that all different environments use the same \emph{getTime()} function and are set with the same clock. This prerequisite is usually easily acquired hence organization can choose to synchronize themselves with external clock services.
%\item Exploit the caller-callee relationship using the method and service name.
%\item Find the entry point; which is the entry service on the target architecture to split our logs into different tree and graph. This step could be either done by mining execution logs and determine the entry point or by defining the \texttt{entry} mark-up within our DSL. 
%\end{itemize}


%\vspace{.15cm}
%
\noindent \textcolor{blue}{\noindent \emph{Transaction Identification.} The goal here is to identify transactions in execution logs. 
\vspace{.15cm}
\\
\noindent A transaction is a time-ordered set of different services and method calls. Within the identified transactions, we focus on the one who contains more than one service.} Indeed, our goal
being to spot poor designs, a single call hardly provides any
useful information about the underlying architecture.
Next, two call tables of transaction are generated at two different layer: service-level and method-level. Thus, whenever a service's method is invoked, the first table would
only register the fact that the service was active whereas the second one
would provide complete details.  Representing transactions on two levels of granularity
helps improve the performances of our detection methods,
e.g. reducing running time and increasing precision. \textcolor{blue}{Indeed, a first analysis is performed at a high granularity level--services--to identify suspicious services; then a deepen analysis is performed on them one using both tables of transaction. }



%\noindent \emph{Step - 2: Construction of transactions.} The goal
%here is to recreate transactions from the plain text execution
%logs.
%
%At this stage, successive logs need to be split
%into transactions. As a general principle, a transaction starts and ends with an action
%on the target SBS. Such actions could either be user-driven
%(service consumption) or generated by others actions.
%Moreover, we assume that relevant
%transactions cover actions with at least two calls. Indeed, our goal
%being to spot poor designs, a single call hardly provides any
%useful information about the underlying architecture.
%Next, two transaction tables are generated, a service-
%and a method-level one. Thus, whenever a service call
%results in two of its methods being invoked, the first table would
%only register the fact that the service was active whereas the second one
%would provide complete details (see Figure~\ref{fig:Service-level-transactions}
%for examples for both levels).  
%%the first one deals with transactions
%%regardless of methods, although if a service invoke two different
%%methods, the first transaction table, will only have the information
%%that this service has been called. In other hand, the second table
%%will take care of methods.
%Representing transactions on two levels of granularity
%helps improve the performances of our detection methods,
%e.g. reducing running time and increasing precision.
%
%
%%By creating two different tables for the
%%two different layers of abstraction (services and methods) we are
%%able to perform pretreatments to optimize our detection algorithms
%%in term of time and precision. Therefore tables contains information
%%like displayed in , the
%%first line represents the service level and the second line, the method
%%level.



\vspace{.15cm}
\noindent \emph{Focus shift.} This feature is the main reason for SOMAD performing better than SODA in the identification of truly harmful SOA antipatterns. \rv{2-6}
\vspace{.15cm}
\\
\noindent Observe that SOMAD hypotheses shift the focus of the antipattern search from pure architectural considerations to usage, thus neglecting the exact values of some basic metrics. It is a natural choice since SOMAD does not access exact values through service interfaces or implementation. Moreover, analyzing a system from the usage view angle should --and this was proven by our experimental study (see below)-- result in a better precision. Consider a service named \emph{Half-Deprecated Service} composed of four methods: A, B, C and D. Assume the methods C and D are outdated yet the service still exposes them to ensure retro-compatibility. One way to compute the cohesion of our service is to count how many of its methods are used during a session by a unique user. Since half of the methods are outdated it is highly probable that any user will consume at most the other half. Therefore, if cohesion is computed from the service interface, it would amount to 0.5 (2/4) which should raise the suspicions of low-cohesion SOA antipatterns. In contrast, if the cohesion is computed from execution traces the result will tend to be 1.0. Indeed, the unforeseen calls to the deprecated methods will most probably be discarded due to their their low support in the execution traces. In summary,  because of its focus on usage, SOMAD should perform better than SODA in detecting harmful SOA antipatterns.
