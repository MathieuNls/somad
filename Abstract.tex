\begin{abstract}

Service Based Systems (SBSs), like other software
systems, evolve due to changes in both user requirements and
execution contexts. Continuous evolution could easily deteriorate
the design and reduce the Quality of Service (QoS) of SBSs and
may result in poor design solutions, commonly known as SOA
antipatterns. SOA antipatterns lead to a reduced maintainability and
reusability of SBSs. It is therefore important to first detect and
then remove them. However, techniques for SOA antipattern
detection are still in their infancy, and there are hardly any tools
for their automatic detection. In this paper, we propose a new
and innovative approach for SOA antipattern detection called SOMAD
(Service Oriented Mining for Antipattern Detection) which is an evolution of the previously published SODA \red{(Service Oriented Detection For Antpatterns)} tool. 
SOMAD improves SOA antipattern detection by mining execution traces:
It detects strong associations between sequences of service/method calls and further filters them using a suite of dedicated metrics. We first present the underlying association mining model and introduce
the SBS-oriented rule metrics. We then describe a validating
application of SOMAD to two independently developed SBSs. A comparison of our new tool with SODA reveals superiority of the former: Its precision is better by a margin ranging from 2.6\% to 16.67\% while the recall remains optimal at 100\% and the speed is significantly reduces (2.5+ times on the same test subjects).

\end{abstract}
\begin{IEEEkeywords}
SOA Antipatterns, Mining Execution Traces, Sequential Association Rules, Service Oriented Architecture.
\end{IEEEkeywords}

